\documentclass[12pt, letterpaper, titlepage]{article}
\usepackage{hyperref}
\usepackage[margin=1in]{geometry}
\hypersetup{colorlinks = true, linkcolor = blue, citecolor=blue, urlcolor = blue}

\title{CSE 5825: Bayesian Machine Learning Project Proposal}
\author{Owen Fiore}

\begin{document}
\maketitle

\section{Overview}
Throughout the NBA's history, players have been divided into five positions:
point guard, shooting guard, small forward, power forward, and center.  However,
now more than ever, players roles are more fluid with players taking on a variety
of offensive and defensive roles regardless of their position.  Thus, there needs
to be a way to analyze a player's contributions to their team, and we can try to
do so by trying to cluster players into known archetypes.  For example, the
point guard has long been responsible for running the team's offense but that
has taken many forms. Steph Curry who is 6'2 and 185 lbs averaged 11.4 three 
point attempts per game and shot 42.7\% on those attempts in 2023.
Tyrese Haliburton who is 6'5 and 185 lbs who averaged 7.2
three point attempts and shot 40.0\% on those attempts but also averaged 4 more
assists per game than Curry.  Steph Curry is primarily focused on scoring while
Haliburton orchestrates his team's offense: averaging the second most assists
in the NBA in 2023, while Curry finished 7th in points per game.  It becomes
clear that despite these players allegedly playing the same position, they play
drastically different roles.  In this project, we are going to try and cluster
players into roles and then try to figure out what combinations of players work
well together.

\section{Hypotheses}
The job of a general manager (GM) of a basketball team is to build a roster of
players that will presently or in the future compete for a championship. Based
on the results above, it seems conducive that there are certain combinations of
players that would work well but also a lot that would work poorly.  Steph Curry
and Tyrese Haliburton play the same position in different ways but would likely
be a poor on the court fit together.  They are both relatively short and may
struggle defensively to guard larger teams.  So if you want to build a roster
around a player like Curry, what other attributes of those players would go well
with Curry's skills?  Curry's two most obvious weaknesses are his defense and his
play making (Curry ranked 23rd in 2023 in assists per game). If you were a GM 
building a team around Curry, you would want players who are strong defensively
and good passers to play alongside him.  If you were building a team around
Haliburton, you may want stronger scorers for him to get the ball to.  Thus, this
is the first objective of this paper, to analyze combinations of players that
work well together to try and figure out what separates championship caliber teams
from other playoff teams that can't break through and collapse in the postseason.

Although a GM's job is to build a competitive roster, there are a lot of constraints
in what they can do, the most obvious of which is salary.  The NBA has a soft
salary cap, which means that although teams can spend as much money as they want,
past a certain point they will start to incur significant penalties which come in
the form of fines and free agency restrictions.  Additionally, the GM has to report
to their boss and must explain the decisions they make, so over-paying for mediocre
players also should be avoided.  In a conversation I have had with one of my
friends, I questioned whether there is more depth at the wing position (SG/SF/PF)
and thus whether a team is better off spending money on positions where talent
is rarer such as at center or point guard.  Some teams such as the Celtics and
Clippers in recent year have spent money to pair two all-star caliber wings and
have essentially neglected other important positions by choosing to spend a
large proportion of their salary on the wing position. Our second goal of this paper is to
take salary into account and investigate what positions GM's should be spending
their money on.

A third and final hypothesis is that certain clusters of players will be much
closer than others.  To a casual NBA fan, it would seem extremely obvious that
all point guards would behave more similarly than that of centers, but one may be
surprised to learn that through certain points of last season, Ja Morant a 6'3
point guard was leading the league in paint scoring, in large part due to him
being an exceptional dunker and utilizing "floaters" in the lane.  Thus, I still
believe this to be a worth hypothesis, as the results from it may be less
predictable than what appears.

\section{Data}
The data primarily comes from Basketball Reference, a website that has
many important statistics already presented in the form of a data frame.  There
will be roughly 50 features that include but are not limited to: three point
percentage, rebounds, assists, usage percent, percent field goal by distance and
field goal percent by distance.  Some features are going to be strong
at separating how different guards play: like above how assists is one metric
that separates Steph Curry and Tyrese Haliburton.  In general, centers average
very few assists, so it doesn't make sense to use assists to compare centers.
Instead, looking at where centers take their shots from can be informative of
their play style and indicate whether they are stretching the court by taking
mid-range or longer shots versus dunks and layups underneath the basket.  Data
for NBA salaries is found online in various capacities, so data frames may be
merged or joined in python's pandas libraries as needed in order to help with
the second hypothesis, which involves conditioning on salary.
\section{Methods}
In order to determine if certain combinations of players work well with each
other, we are going to need a two-step process.  The first is going to be a
clustering step: where we try to combine individuals of certain attributes based on
heuristics that match how they play.  And the second where we use those clusters
and apply some Bayesian sampling techniques and try to answer the
hypothesis questions presented above.  I ideally would implement representational
learning to try and cluster players into the archetypes.  I am still exploring
if I want to try and cluster by position or abandon this approach.  In the modern
NBA, shooting guards, small forwards and power forwards are typically deployed
to a variety of roles depending on the other players (both on their team and
the opponent's) on the court.  However, we could increase the number of clusters
from five to nine (include the four combinations of players such as small
forward/power forward, etc.) to try and see if this clusters better.  Once the
clustering is complete, simulation and modeling will take place to try and predict
what combinations of players make a team successful.  I do not know yet what I
will use, but possibly a mixed model in combination with an expectation
maximization algorithm to iteratively find optimal combinations from clusters.
I think that the more important part of this will be the first step, because the
results may be largely inconclusive or uninformative if the first part is not
successful, and therefore I want to make sure that I get that part right first.
\end{document}