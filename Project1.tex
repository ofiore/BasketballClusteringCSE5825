\documentclass[12pt, letterpaper, titlepage]{article}
\usepackage{hyperref}
\usepackage[margin=1in]{geometry}
\hypersetup{colorlinks = true, linkcolor = blue, citecolor=blue, urlcolor = blue}

\title{CSE 5825: Bayesian Machine Learning Project Proposal}
\author{Owen Fiore and Jared Sullivan}

\begin{document}
\maketitle

\section{Working with data}
\subsection{Question 1}
We have collected data from Basketball Reference and was converted into csv
format. Data currently is from the last five seasons of the NBA: 2019-2023 however
we may include more data from previous years as appropriate.  We chose recent
data due to concerns about changing of league strategies and overall change
in how basketball is played in the NBA.  More now that ever, players have tried
to vary their skill set to be more versatile and thus adaptable to other teams.
The data is fairly exhaustive, as it contains every important statistic tracked
in basketball in addition to many advanced statistics that have been developed
in recent years by basketball statisticians to try and gauge player performance.
In terms of biases of the data, there may be issues with how recent it is, and
we may want to include more data to increase sample size.  There are concerns
about the quality of data from 2020 due to the impact of the NBA Bubble which
was instituted in Orlando, Florida as a result of the COVID-19 epidemic which
occurred towards the end of the 2020 NBA regular season.  However, the Bubble
should hopefully not significantly impact the role of players in relation to
their team.

\section{Thinking about models}
\subsection{Question 1}
The observed variables in the data are player statistics that include shooting
percentages, how efficient players were, defensive contributions, and where
players shot from.  There are many columns of the data that are calculated from
other columns, but we chose to include these as they are informative.  For
example, field goal percent is the player's made field goals divided by their
field goal attempts.  Looking at field goal percent by itself can be helpful but
there are players in the NBA who may have high field goal percentages because
they do not shoot a lot and when they do, their shots are mostly dunks which are
generally made shots.  Essentially they do not attempt many shots and the shots
they do attempt are easy to make.

\subsection{Question 2}
One hidden variable we are going to implement is a scaled position estimate. We
have the estimates of the percent at each position a player played during the
season.  Using this we can create a scale from 1 to 5 of the five positions.  For
example a player who is at 1.00 is a pure point guard, 5.00 is a pure center,
but 1.50 is a player halfway between point guard and shooting guard.  We are able
to do this because basketball player positions are ordinal.  Based on this we
can explore trends and see if some players 

\end{document}